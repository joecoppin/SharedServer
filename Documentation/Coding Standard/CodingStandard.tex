\documentclass{article}
\usepackage[utf8]{inputenc}
\usepackage[english]{babel}

\usepackage[dvipsnames]{xcolor}
\usepackage{minted}

\definecolor{LightGray}{gray}{0.9}

\title{Coding Standard}
\author{michael.kamerath }
\date{October 2016}

\begin{document}

\maketitle

\renewcommand\listoflistingscaption{List of code examples}
\listoflistings

\newpage\section{Clang}
All code SHALL be formatted with clang format with the .clang\_format file found in the repository. \\ \\
\textit{Rationale: This ensures that all code has similar white spacing and is therefore, easy to read.}

\section{Naming Conventions}
\subsection{Variable Names}
Global variable SHALL be prefaced with g\_ \\
Static variables SHALL be prefaced with s\_ \\
Pointers SHALL be preceded by a p \\
Non Const references shall be preceded by an r \\
Combinations of the above attributes can happen and both prefaces SHALL be accounted for \\
\textit{Rationale: Consistent naming conventions ensures that proper care is taken for special cases, ie static variables and pointers.} \\

\begin{listing}[ht]
\begin{minted}[bgcolor=LightGray]{cpp}
int g_someGlobal = 10;
int gs_someStaticGlobal = 10;
int someFunction()
{
    static int s_num = 0;
    return ++s_num;
}
int main()
{
    std::unique_ptr<MyClass> pClassPtr(new MyClass);
}
\end{minted}
\caption{Variable Naming}
\end{listing}

\newpage\subsection{Enums and Constants}
Enum values and compile time constants SHALL be in all capital letters. \\
\textit{Rationale: All caps lets developers know a value is immutable which helps in debugging purposes} \\

\begin{listing}[ht]
\begin{minted}[bgcolor=LightGray]{cpp}
const int SIZE = 100;
enum Color
{
    RED,
    ORANGE,
    YELLOW
};
\end{minted}
\caption{All Caps}
\end{listing}

\subsection{User Defined Types}
\noindent Enums, Structs, Classes, and Typedefs SHALL be PascalCase. \\
\textit{Rationale: PascalCase allows for easy identification of user defined types.} \\

\begin{listing}[ht]
\begin{minted}[bgcolor=LightGray]{cpp}
enum AllTheColors
{
};
struct MyStruct
{
};
class MyClass
{
};
typdef unsigned long long UInt64;
\end{minted}
\caption{User Defined Types}
\end{listing}

\newpage\subsection{Primitive Data Types}
\noindent variables and function names SHALL be camelCase. \\
\textit{Rationale: camelCase allows for easy identification of user defined types.} \\

\begin{listing}[ht]
\begin{minted}[bgcolor=LightGray]{cpp}
int someFunction()
{
    std::string firstName;
    int value = 0;
    return value;
}

\end{minted}
\caption{Primitive Data Types}
\end{listing}

\section{Logic}
Logic errors SHALL be fixed before code is merged. It is not reasonable to enumerate all of these practices. Guides for industry best practices in C++ include \\
\begin{enumerate}
    \item \textit{Effective C++: Scott Meyers}
    \item \textit{Effective Modern C++: Scott Meyers}
    \item \textit{Exceptional C++: Herb Sutter}
\end{enumerate}
\textit{Rationale: The books listed above are considered by many to be industry standard and give a good overview of good practices.}

\section{Comments}
Functions should be named in a way that they can be easily understood by the name. Any function with at least one parameter SHALL have a brief comment saying what the function does. Any part of a block of code that is hard to follow SHOULD be prefaced with TRICKY: with an explanation on why the code is that way. \\
\textit{Rationale: Although aptly named functions are preferred, comments help to communicate developer intent.}

\newpage\section{Initialization}
\subsection{Class Member Initialization}
All variables SHALL be initialized, member variables SHALL be initialized in an initialization list. \\
\textit{Rationale: Uninitialized variables can lead to unexpected errors that are hard to debug.}

\begin{listing}[ht]
\begin{minted}[bgcolor=LightGray]{cpp}
int num = 0;
class Car
{
    private:
    string model;
    string make;
    int year;
    
    Car() :
    model(""),
    make(""),
    year(1900)
    {
    }
};
\end{minted}
\caption{Class Member Lists}
\end{listing}

\newpage\noindent
\subsection{RAII}
Use the RAII (Resource Acquisition is Initialization) idiom for resource management. Pointers SHALL be handled in this way with the exception of wxWidgets. Mutexes SHALL be handled by lock\_guard rather than mutex.lock() \\
\textit{Rationale: If an exception is ever thrown, poorly managed memory can be leaked causing the state of a program to become unstable.} \\

\begin{listing}[ht]
\begin{minted}[bgcolor=LightGray]{cpp}
std::mutex mx;
void problem()
{
    mx.lock();          //Acquires the mutex
    functionCall();     //If this function throws an exception,
                        //the mutex is never released.
    if (someCondition)
    {
        return 0;       //If the condition is met and the
                        //function returns early, the
                        //mutex is never released.
    }
    mx.unlock();        //Release the mutex
}

void RAII()
{
    std::lock_guard<std::mutex> lck(mx); //RAII: mutex
                                         //acquisition is
                                         //Initialization
    
    functionCall()                       //If this throws an 
                                         //exception, the 
                                         //mutex is released
    
    if (someCondition)
    {
        return 0;       //Early return statement if fine.
                        //The mutex is released
    }
}
\end{minted}
\caption{RAII}
\end{listing}



\end{document}
